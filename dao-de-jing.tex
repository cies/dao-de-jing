% Resume of Cies Breijs
% =====================

% To generate a pdf from this file in Ubuntu:
%   sudo aptitude install texlive-xetex tex-gyre texlive-latex-recommended
%   xelatex cies-breijs-resume.tex
%
% I don't use pdflatex but XeTeX since it allows me to use gyre's lowercase numbers.

% vim:set ft=tex:

\documentclass[10pt,a4paper]{book}
\usepackage[a4paper,margin=0.75in]{geometry}
\usepackage[utf8]{inputenc}
\usepackage{xeCJK}
\usepackage{mdwlist}
\usepackage{multicol}
%\usepackage[T1]{fontenc}  % not working with xeCJK
\usepackage{textcomp}
\usepackage{relsize}  % for \textscale, which I prefer over \sc (small caps), see my \acr command

\usepackage[pdftex]{hyperref}  % yups, URLs everwhere...
\usepackage{xcolor}  % ... and color them links
\definecolor{dark-blue}{rgb}{0.15,0.15,0.4}
\hypersetup{colorlinks,linkcolor={dark-blue},citecolor={dark-blue},urlcolor={dark-blue}}

\newfontfamily\zhpunctfont[RawFeature={vertical:+vert:+vhal}]{AR PL UKai TW}

% \usepackage{tgpagella}  % the pretty fonts
% \usepackage{fontspec}
% \addfontfeature{Style=Historic}
% \setmainfont
%   [ ExternalLocation ,
%     Mapping = tex-text ,
%     Numbers = OldStyle ,
%     Ligatures= {Common,Historical,Contextual,Rare} ,
%     BoldFont = texgyrepagella-bold.otf ,
%     ItalicFont = texgyrepagella-italic.otf ,
%     BoldItalicFont = texgyrepagella-bolditalic.otf ]
%   {texgyrepagella-regular.otf}

% use this for the libertine font (has more lignatures), install 'ttf-linux-libertine' on ubuntu
% \setmainfont
%   [ ExternalLocation = /usr/share/fonts/truetype/linux-libertine/ ,
%     Mapping = tex-text ,
%     Numbers = OldStyle ,
%     Ligatures= {Common,Historical,Contextual,Rare} ,
%     BoldFont = LinLibertine_Bd.ttf ,
%     ItalicFont = LinLibertine_It.ttf ,
%     BoldItalicFont = LinLibertine_BI.ttf ]
%   {LinLibertine_Re.ttf}

% \setCJKmainfont[RawFeature={vertical:+vert:+vhal}]{AR PL UKai TW}
\setCJKmainfont{AR PL UKai TW}


\begin{document}

\frontmatter


\chapter*{道德經 Dao De Jing}
\thispagestyle{empty}

\section*{\huge \center Translated freely by Cies Breijs}

\newpage
(based on other translations)

In this translation I try to reflect my understanding of the text.
So what I don't understand I will not translate (see the [...]).


\mainmatter

\begin{multicols}{2}
{\huge Chapter 1}
\vspace{15pt}
\\
Dao expressed in words in not eternal Dao;\\
as words, or names, change meaning over time.\\
\\
Nameless is the origin of Heaven and Earth.\\
By naming we create our own world.\\
\\
Without desire we can see the Nameless beyond words;\\
as from desire we create our own world by our names.\\
\\
Nameless or named, both have the same origin:\\
the original emptiness,\\
the emptiness before creation,\\
the gate to wisdom.\\

\columnbreak

\setRL
{\huge 第一章}
\vspace{15pt}
\\
道可道,非常道。名可名,非常名。\\
無名天地之始﹔有名萬物之母。\\
故常無,欲以觀其妙﹔常有,欲以觀其徼。\\
此兩者,同出而異名,同謂之玄。\\
玄之又玄,眾妙之門。
\end{multicols}


2.
By naming beauty, we give rise to ugly.
We know what is good because we know evil.
When we posess we can loose.
Difficult helps us to see easy.
Long is long because there is short.
High and low need eachother.
After comes only after before.

Those who understand this:
they act altruisticly,
they teach the Nameless without naming it,
they see creations come and go, but hold on to none.

Their actions are desireless, and
they do not claim their work, yet
their work stands the test of time.


3.
Not to brag about what is great,
to avoid disputes.
Not to show off what is worthy,
to avoid stealing.
Not to give rise to desire,
to avoid confusion.

The Wise lead by:
clearing people's minds and growing their hearts,
reducing people's ambitions and strengthening their endurance.

They show us a way beyond knowledge and desire.
Be careful not to desire knowledge of this way!
By acting naturally and without desire,
we do not need to worrie about the way.


4.
Dao is like emptiness:
used a lot, but never used up.
hidden for the senses, but ever present.

I do not know its origin, it has just always been around.


5.
Dao is impartial,
good and evil are simply names.
The Wise are impartial,
they treat people from one intention.

Dao, like emptiness, is infinitely capable.
The more we use it, the more is works for us.
The more we talk of it, the less we understand it.
We find it when holding firm to our center.


6.
Dao is called the Great Mother:
empty yet inexhaustible,
She gives birth to infinite creations.

Not too obvious,
yet ever present,
inside us,
for us to use,
anytime.


7.
Dao is eternal.
Unborn and not existing just for itself:
therefor eternal.

The Wise:
hold themselves back, thereby take the lead.
being detached, they are one with all.
through selflessness, they accomplish.


8.
The Truth is like water.
It benefits innumerable creatures,
yet never over-worked.
It flows naturally,
without clinging to its surroundings.
So does Dao.

Keep the feet on the ground.
Carry simple thoughts and a humble heart.
Be gentle to others.
Speak truthfully.
Rule just.
Be competent in everyday life.
Act timely.
Do not compare or compete with others.

It will work.


9.
Fill it up to the rim,
and some might get spilled.
Very sharp knifes,
easily go blunt.
Showing off wealth,
disturbs a simple mind.

Act without being attached to the result:
that is the way of Heaven.


10.
Can you control your thoughts,
and keep your mind impartial?

Can you allign body and mind,
and become as innocent as a newborn?

Can you clear your conscience,
and use it as your guide?

Can you love people and lead them,
without imposing your will?

Can you deal with highly important matters,
and let it happen naturally?

Can you forget what you know,
and understand what is important?

Giving birth and nourishing,
having without holding on to it,
working without expecting a reward,
leading without giving orders.

This is true Virtue.


11.
A wheel's spokes connect at the hub,
the center hole makes it useful.
A ceramic pot is shaped from clay,
the hollow belly makes it useful.
A block-hut is build with wood,
the space inside makes is useful.

Properties stem from existance,
but usefulness stems from emptiness.


12.
Colors blind the eye.
Tones deafen the ear.
Flavors dull the taste.

Racing and hunting madden the mind.
Precious things lead astray.

Therefore,
the Wise is guided by what he feels,
instead of what he sees.

How?
He just lets go of somethings,
while holding on to something.


13.
Status and disgrace cause fear.
Our greatest worries derive from our body.

What is meant by:
  Status and disgrace cause fear.

Obtaining status causes fear,
as loosing it is painful.
This implies status and disgrace cause fear.

What is meant by:
  Our greatest worries derive from our body.

If the emptiness of self is not realized.
One experiences great worries,
as one identifies with the body.
Without a body, what could cause worries?

One who regards selflessly,
serves all below Heaven,
and is fit to be entrusted with the world.

One who loves selflessly,
serves all below Heaven,
and it fit to care for the world.


14.
Looking at it without seeing it: formless.
Listening for it without hearing it: silent.
Reaching for it without grabbing it: intangible.

These three show it is beyond definition.
Since they blend together and act as one.

No up-side and no down-side.
Formless and unimaginable.
Subtle and trancending all conception.

An unbroken thread beyond description,
leading back to nothingness.

Approaching it, there is no beginning.
Following it, there is no end.

Practise this wordless teaching,
and move with the present.
Serve the original conscience: embody Dao.


15.
The Wise of the past where subtle, mysterious, profound and responsive.
Their understanding so deep: impossible to put in words.
Therefor words are used merely to describe their appearance.

Watchful, as someone on slippery ice.
Alert, as a soldier in enemy lines.
Respectful, as a visiting guest.
Yielding, as melting ice.
Simple, as a log of wood.
Receptive, as a valley.
Opaque, like a muddy pool.

Who can wait quietly for mud to settle in otherwise clear water?
Who can remain still allowing right action to arise by itself?

Alligned with Dao one will not seek fullfillment.
Without seeking fullfillment we are naturally fullfilled.


16.
Without clinging to concepts,
heart and mind find peace.
Innumerable creations rise and fall,
spawning from and returning back to their origin.
Everything returns back nothing.
Nothing, the common source of everything.

Without realizing the source,
things may seem confusing.

Realizing our common origin opens up our mind.
An open mind allows a kind heart.
A kind heart alligns with Dao.

Alligned with Dao, we live at the source -- the origin of all.
Upon death we need not to leave nothing behind.


17.
Unnoticed the Wise take lead.
Great leaders are loved by many.
Hated are those who rule with fear.

Trusting and respecting others,
will earn their trust and respect.

The Wise do not give orders,
they work along with the people.
When the Wise accomplish,
the people are proud of themselves.


18.
When mankind left Dao to pursue its own ways,
words like `kindness' and `morality' arose.

It is in the absense of wisdom that,
cleverness and knowledge took over.

Without harmony in a family,
filial piety gets its name.

When a nation is unstable,
patriotism is born.


19.
Get rid of `holiness' and `wisdom', for
people will understand the truth intuitively.

Get rid of `kindness' and `morality', for
people will naturally do the right thing.

Get rid of `value' and `profit', for
people will not fall into stealing.

These three merely describe outward forms,
incapable of making any difference.

More important:
to see the simplicity,
to realize ones True Nature,
to let go of selfishness, and
to temper desire.


20.
Stop selfish thinking, for it just causes trouble.

What is it that sets `yes' and `no' apart?
Who defines the difference between success and failure?
Why value what others value and avoid what others avoid?

When people enjoy, celebrate and party,
it seems like I am the only one not excited.
In contrast to them I seem boring.
But a baby before it can smile can still be happy.

When people have what they need,
it seems I have nothing at all.
In contrast to them I just drift about.
My mind is so empty one could call me stupid.

When people are bright, I am dark.
When people are witfull, I am dull.
When people are clever, I have no clue.

Drifting randomly, like waves on sea.
Going without a goal, like the wind.

I am different from sensient beings:
I dwell on the Great Mother.


21.
The Wise keep their mind aligned with Dao,
this alone makes them wise.

Dao can never be understood.
So how do the Wise align with it?
They simply do not cling to ideas.

Dao is too dim for anyone to see,
So how can it make a person wise?
They simply do not stand in its way.

Before time an space, there was Dao,
it is beyond birth and death.
So how can one know this is true?
Just look inside and see.


22.
Give up to overcome.
Be flexible in being straight.
Empty to be fulfilled.
Wear out to renew.
Renounce worldly and be given the world.

The Wise, allingned with Dao, set an example for people.
Unadvertised, people still find them.
Having nothing to prove, people can trust their words.
Through selflessness people can recognize themselves in them.
Without arguments they remain undisputed.
With no goal on their mind, they succeed in everything.

It is said: let go to overcome.

Is that an empty saying?
Do `to let go' and `to overcome' not mean the same thing?

It is not through words, but though allignment with Dao,
that true understanding arises.


23. *have more looks at this one
To talk little is natural: just say what you have to say.

High winds do not last all morning,
and heavy rains do not last all day.
If Heaven and Earth cannot make wild things last,
how could it be possible for man?

Who is alligning with Dao, is Dao.
Who is embodying virtue, is virtuous.
Who is losing the way, is lost.

Being Dao, ones Path will True.
Being virtuous, virtue is with you.
Being lost, is a choise.

Who cannot have trust, will never be trusted.


24.
Standing on tiptoe we cannot stand firm.
Always running around will not carry us far.
Trying to shrine bright only masks our light.
Defining ourselves keeps us from seeing who we are.
By showing off we merely show we have no merits.
Driven by the fruit of our actions we create nothing that last.

Alligned with Dao one understands these and avoids them.
Like avoiding sickness by not exposing ones body too much.


25.
Before all creation we find something formless and perfect.
Serene, empty, solitary, unchanging, infinite and eternally present.

It is the mother of the universe.
I do not know its name -- I call it Dao.

It flows through all things, inside and outside,
and returns to the origin of all things.

Dao is great.
Heaven is great.
Earth is great.
Man is great.

The great four of the universe.

Man follows Earth.
Earth follows Heaven.
Heaven follows the Dao.
Dao follows only itself.


26.
Steadfast is the root of confusion.
Still is the origin of the restless.

No matter how far from home the Wise stays put at his origin.
However distracting the views, he remains serenely at one.
Can a ruler behave himself frivolous in public?

When confused one breaks with the root.
When restless one moves away from the orgin.


27.
%[first line represents the first few]

Great personalities are build by simple, thorrough deeds.

Therefore, the Wise...
assists all people, discarding no-one;
cares for all things, abandoning nothing.

This is called: following the inner light.

The more advanced teach those on behind,
thereby allowing further advancement of their teachers.

If teachers are not respected,
and those on behind not cared for,
then confusion will arise:
confusion that cannot be avoided by cleverness.

This is an essential principle.


28.
Know strength, yet stay gentle. Be given the world.
Upon receiving the world, Dao will never leave you,
and you will be innocent like a new born.

Know light, yet stay in the shade. Be an example for the world.
As an example for the world, Dao will be strong inside you,
and there will be nothing you cannot do.

Know honour, yet stay humble. Accept the world as it is.
Accepting the world for what it is, Dao will be luminous inside you,
and you will return to your primal self.

The world is created from the void,
like utensils are carved out of a block of wood.
The Wise knows the utensils,
yet keeps to the the block:
thus she can use all things.
%==  seek some guidance here ^^ and vv
When the block is carved, it becomes useful.
When the sage uses it, he becomes the ruler.
Thus, "A great tailor cuts little."


29.
One trying to improve the universe will not succeed.
The universe is sacred, one cannot improve it.
Who tries to tamper with it, fools himself.
Who tries to hold on to it, loses it.

Hence, there is a time to go ahead and a time to stay behind.
There is a time to breathe easy and a time to breathe hard.
There is a time to be vigorous and a time to be gentle.
There is a time to gather and a time to release.

Therefore the Wise avoids extremes, self-indulgence and extravagance.


\backmatter

\begin{thebibliography}{99}

\bibitem{baba_books}
Books of Shrii Shrii Anandamurti (Prabhat Ranjan Sarkar): \\
http://shop.anandamarga.org/

\end{thebibliography}

\end{document}
