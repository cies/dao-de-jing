\documentclass{article}
\usepackage{genzi}
\usepackage{xltxtra}
\title{Human greed as depicted in Rasyoumon (羅生門)}
\author{Kazuomi Kuniyoshi (國吉一臣)}
\date{\today}
\begin{document}
\maketitle
\section*{Introduction}
Ryûnosuke Akutagawa (芥川龍之介) opens his short novel \textit{Rasyoumon} (羅生門) as follows:
\begin{quote}
或日の暮方の事である。一人の下人が、羅生門の下で雨やみを待つてゐた。
\end{quote}
At the beginning of the third paragraph ‘何故かと云ふと、この二三年、京都には’, he tells the reader explicitly that the story is situated in Kyôto. Now let me quote the beginning of the second poem from \textit{Kalevala}:
\begin{verse}
Nousi siitä Väinämöinen\\
jalan kahen kankahalle\\
saarehen selällisehen,\\
manterehen puuttomahan.
\end{verse}
Let me turn to the beginning part of the fourth paragraph of \textit{Rasyoumon}:
\begin{quote}
\begin{ja}
その代り又鴉が何處からか、たくさん集つて來た。晝間見ると、その鴉が何羽となく輪を描いて高い鴟尾のまはりを啼きながら、飛びまはつてゐる。
\end{ja}
\end{quote}

As the reader may well be aware by now, this is a dummy article to show what you can do with \XeLaTeX\ and the \textit{Genzi} package, and the text does not make any sense. But this nonsense article ends with a poem which makes sense (\textit{Dhammapada}, 103).
\begin{verse}
{\fontspec[Scale=0.95]{Gentium Basic}Yo sahassaṁ sahassena,\\saṅgāme mānuse jine;\\Ekañca jeyyamattānaṁ,\\sa ve saṅgāmajuttamo.}

Though he should conquer\\
A thousand thousand men in battle,\\
Yet he is the noblest victor\\
Who should conquer himself.\\
\quad\quad (Traslation by Peter Harvey, \textit{An Introduction to Buddhist Ethics})

戦場において百万人に勝つよりも、\\
唯だ一つの自己に克つものこそ、\\
実に最上の勝利者である。\\
\quad\quad(中村元訳「ブッダの真理のことば・感嘆のことば」より)
\end{verse}
\end{document}
